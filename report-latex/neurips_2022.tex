\documentclass{article}


% if you need to pass options to natbib, use, e.g.:
%     \PassOptionsToPackage{numbers, compress}{natbib}
% before loading neurips_2022


% ready for submission
%\usepackage{neurips_2022}


% to compile a preprint version, e.g., for submission to arXiv, add add the
% [preprint] option:
\usepackage[preprint, nonatbib]{neurips_2022}


% to compile a camera-ready version, add the [final] option, e.g.:
%     \usepackage[final]{neurips_2022}


% to avoid loading the natbib package, add option nonatbib:
%    \usepackage[nonatbib]{neurips_2022}


\usepackage[utf8]{inputenc} % allow utf-8 input
\usepackage[T1]{fontenc}    % use 8-bit T1 fonts
\usepackage{hyperref}       % hyperlinks
\usepackage{url}            % simple URL typesetting
\usepackage{booktabs}       % professional-quality tables
\usepackage{amsfonts}       % blackboard math symbols
\usepackage{nicefrac}       % compact symbols for 1/2, etc.
\usepackage{microtype}      % microtypography
\usepackage{xcolor}         % colors

\usepackage{biblatex}
\addbibresource{neurips_2022.bib}
\usepackage{pythonhighlight}

\title{Computer Vision Proseminar Report WS2023}


% The \author macro works with any number of authors. There are two commands
% used to separate the names and addresses of multiple authors: \And and \AND.
%
% Using \And between authors leaves it to LaTeX to determine where to break the
% lines. Using \AND forces a line break at that point. So, if LaTeX puts 3 of 4
% authors names on the first line, and the last on the second line, try using
% \AND instead of \And before the third author name.


\author{%
    Markus Diller\\
    University of Salzburg\\
    Salzburg, 5020 \\
    \texttt{markus.diller@stud.plus.ac.at} \\
    \And
    Marcel Sargant \\
    University of Salzburg \\
    Salzburg, 5020 \\
    \texttt{marcel.sargant@stud.plus.ac.at} \\
}


\begin{document}


    \maketitle


    \begin{abstract}
        The abstract paragraph should be indented \nicefrac{1}{2}~inch (3~picas) on both the left- and right-hand margins.
        Use 10~point type, with a vertical spacing (leading) of 11~points.
        The word \textbf{Abstract} must be centered, bold, and in point size 12.
        Two line spaces precede the abstract.
        The abstract must be limited to one paragraph.
    \end{abstract}


    \section{Introduction}\label{sec:introduction}


    \section{ConvNeXt}\label{sec:convnext}
    ConvNeXt\cite{liu2022convnet}.


    \section{Documentation}\label{sec:documentation}
    For our work with the ConvNeXt architecture we investigated three main ideas.
    The first and most thorough was the effect of data augmentation on the network performance.
    The second was to experiment with the finetuning provided by the given implementation.
    Finally, the third focus was to revert some of the changes \Citeauthor{liu2022convnet} made to the ResNet they built upon.

    \subsection{Code Modification and Quality of Life}\label{subsec:code-modification}
    To allow for an improved workflow for testing we chose to make some minor changes to the given codebase.
    This was done to not only ensure faster training times, but also for a more lenient experience when having to set up the environment.

    The first and only change that meddles with the actual implementation was to add another command line argument to allow for downsampling CIFAR10.
    For this we registered a \texttt{--downsample} argument to the used argparser that is then used in the \verb|build_dataset| function in the \texttt{dataset.py} file.
    The implementation for the usage can be seen in~\ref{fig:downsampling}.
    Instead of choosing a random subset we choose every $n$th element, where $n$ is the downsampling factor.
    Additionally, for transparency, the original codebase used CIFAR100, but we chose to change it to CIFAR10.
    \begin{figure}[h]
        \begin{python}
dataset = datasets.CIFAR10(args.data_path, train=is_train,
                                transform=transform, download=True)
sample = list(range(0, len(dataset), args.downsample))
dataset = torch.utils.data.Subset(dataset, sample)
nb_classes = 10
        \end{python}
        \caption{Downsampling the dataset}
        \label{fig:downsampling}


    \end{figure}

    The second and final change was to create a \texttt{requirements.txt} file for easy installation of the necessary packages.
    Due to the codebase being incompatible with versions of pytorch larger than 2.0 and Google Colab defaulting to such a version we created this file to easily install all necessary dependencies with a single command.
    This comes especially in handy because Colab resets the runtime after just a few hours of inactivity, in turn also resetting the changes in installed packages.
    It eludes us as to why there was no such file already included in the codebase, just a very bare-bones markdown file with requirements.

    \subsection{Data Augmentation}\label{subsec:data-augmentation}

    \subsection{Finetuning}\label{subsec:finetuning}

    \subsection{Network Reversion}\label{subsec:network-reversion}


    \section{Conclusion}\label{sec:conclusion}

    \printbibliography
%%%%%%%%%%%%%%%%%%%%%%%%%%%%%%%%%%%%%%%%%%%%%%%%%%%%%%%%%%%%
    \section*{Checklist}
     \begin{enumerate}


        \item For all authors\ldots
        \begin{enumerate}
            \item Do the main claims made in the abstract and introduction accurately reflect the paper's contributions and scope?
            \answerYes{}
            \item Did you describe the limitations of your work?
            \answerYes{}
            \item Did you discuss any potential negative societal impacts of your work?
            \answerNo{}
            \item Have you read the ethics review guidelines and ensured that your paper conforms to them?
            \answerNo{}
        \end{enumerate}


        \item If you are including theoretical results\ldots
        \begin{enumerate}
            \item Did you state the full set of assumptions of all theoretical results?
            \answerNA{}
            \item Did you include complete proofs of all theoretical results?
            \answerNA{}
        \end{enumerate}


        \item If you ran experiments\ldots
        \begin{enumerate}
            \item Did you include the code, data, and instructions needed to reproduce the main experimental results (either in the supplemental material or as a URL)?
            \answerYes{}
            \item Did you specify all the training details (e.g., data splits, hyperparameters, how they were chosen)?
            \answerYes{}
            \item Did you report error bars (e.g., with respect to the random seed after running experiments multiple times)?
            \answerNA{}
            \item Did you include the total amount of compute and the type of resources used (e.g., type of GPUs, internal cluster, or cloud provider)?
            \answerYes{}
        \end{enumerate}


        \item If you are using existing assets (e.g., code, data, models) or curating/releasing new assets\ldots
        \begin{enumerate}
            \item If your work uses existing assets, did you cite the creators?
            \answerYes{}
            \item Did you mention the license of the assets?
            \answerYes{}
            \item Did you include any new assets either in the supplemental material or as a URL?
            \answerNo{}
            \item Did you discuss whether and how consent was obtained from people whose data you're using/curating?
            \answerNo{}
            \item Did you discuss whether the data you are using/curating contains personally identifiable information or offensive content?
            \answerNo{}
        \end{enumerate}


        \item If you used crowdsourcing or conducted research with human subjects\ldots
        \begin{enumerate}
            \item Did you include the full text of instructions given to participants and screenshots, if applicable?
            \answerNA{}
            \item Did you describe any potential participant risks, with links to Institutional Review Board (IRB) approvals, if applicable?
            \answerNA{}
            \item Did you include the estimated hourly wage paid to participants and the total amount spent on participant compensation?
            \answerNA{}
        \end{enumerate}


    \end{enumerate}

\end{document}